\documentclass{article}
\usepackage{ctex}
\usepackage{setspace}
\usepackage{graphicx}
\usepackage{amsmath}
\usepackage{textcomp}
\usepackage{amssymb} 
\usepackage{amsfonts}
\usepackage{amsfonts}
\usepackage{esint}
\usepackage{bm}
\usepackage{geometry}
\usepackage{fancyhdr}
\usepackage{siunitx}
\usepackage{wrapfig}
\usepackage{tikz}
\begin{document}

\title{ \linespread{1.9}\selectfont 
\Huge{Neural Network and Deep Learning}}
    \author{Kuang Yaming Honors School\\171240510~~Ma Yuheng}
    \date{2019.8.28}
    \maketitle
In the unzipped package, five files are presented. 
\begin{itemize}
\item mnist.pkl.gz, contains all MNIST data we will use
\item mnist\_loader.py, for reading the MNIST data
\item network.py, the main work we have done for network training
\item exe.py, execution of training and digits recognition
\item 1.png, an example picture of 0
\end{itemize}
\par
In this document, I will present what I did based on knowledge from summer school, basically as following:
\begin{itemize}
\item apply stochastic gradient descent for feedforward network, which makes the SGD self-adjustable
\item apply cross-entropy method to avoid slowing down when outputs greatly differs from expectation
\item apply regularization to cost function to avoid overfitting as well as uncommon large weights
\item apply adjusted initialize way to weights to prevent hidden layers to be saturated
\item figure out the structure of MNIST and write a program to read and classify input hand-written digit
\end{itemize}
\par
I have to admit that the codes are directly copied from Github of Michal Daniel Dobrzanski\footnote{https://github.com/MichalDanielDobrzanski}. I read through it and made some adjustments for detail. The first four improvements are talked about in chap. 3 of the book. The codes are self-descriptive and there are fairly comments in the line so I am not going to explain much here. What I want to do is to show that I can recognize a hand-written digit by myself. The process needs the third-party library $\textbf{cv2}$, which can be acquired by entering $pip install opencv-python$ in console( if you have installed pip) or follow the website here\footnote{https://pypi.org/project/opencv-python/}.\par
Open exe.py, and assign the index of folder you unzipped the package onto index in console, for example:
$$ \$\;\;\; index='/Users/mayuheng/Desktop/deeplearning'$$
then run the file. As you can see the results coming out, the final accurancy on training data is about 97.664\% and we are confident that overfitting barely happened because this percentage keeps growing in 30 epoch trainings. After the training results, I have the recognition answer of the self-given picture. You can replace it to verify but please write clearly because it's about my grades of summer school.
\end{document}