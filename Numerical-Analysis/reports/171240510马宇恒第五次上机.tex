% !Mode:: "TeX:UTF-8"

\documentclass[a4paper,11pt,onecolumn,twoside]{article}
\usepackage{fancyhdr}
\usepackage{amsmath,amsfonts,amssymb,amsthm}
\usepackage{graphicx}
\usepackage{verbatim}
\usepackage{mathptmx}
\usepackage{booktabs}
\usepackage[labelfont=bf]{caption}
\usepackage{indentfirst}
\usepackage{caption}
\usepackage{setspace}
\usepackage{enumitem}
\usepackage{subfigure}
\usepackage{fontspec}
\usepackage{appendix}
\usepackage{listings}
\usepackage{xeCJK}
\newtheorem{lemma}{引理}[section]% Please change the following fonts if they are not available.
\setmainfont{Times New Roman}

\addtolength{\topmargin}{-54pt}
\setlength{\oddsidemargin}{-0.9cm}
\setlength{\evensidemargin}{\oddsidemargin}
\setlength{\textwidth}{17.00cm}
\setlength{\textheight}{24.50cm}

\renewcommand{\baselinestretch}{1.1}
\parindent 22pt

\title{\textbf{第五次上机实验报告}}
\author{
马宇恒
\\[2pt]
{\small \textit{匡亚明学院 171240510}}}
\date{}
\fancypagestyle{firststyle}
{
   \fancyhf{}
   \fancyhead[C]{数值计算实验报告}
   \fancyhead[R]{\thepage}
}

\pagestyle{fancy}
\fancyhf{}

\fancyhead[LE,RO]{\thepage}
\fancyhead[CE]{数值计算与实验}
\fancyhead[RE]{}
\fancyhead[CO]{}
\fancyhead[LO]{}

\setlist{nolistsep}
\captionsetup{font=small}

\newcommand{\supercite}[1]{\textsuperscript{\cite{#1}}}
\begin{document}
\maketitle
\thispagestyle{firststyle}
\setlength{\oddsidemargin}{ 1cm}
\setlength{\evensidemargin}{\oddsidemargin}
\setlength{\textwidth}{15.50cm}
\vspace{-.8cm}


\setcounter{page}{1}

\setlength{\oddsidemargin}{-.5cm}  % 3.17cm - 1 inch
\setlength{\evensidemargin}{\oddsidemargin}
\setlength{\textwidth}{17.00cm}

\section{实验目的}
实现高斯列主元、按比例列主元、非交换按比例列主元、高斯约旦消元法解线性方程组。并使用他们分别求解例题和以希尔伯特矩阵作为增广矩阵的方程组。

\section{任务}
\subsection{算法}
从基本的高斯消元思路出发,通过使用数组分别保存行主元绝对值、行交换顺序,改变消元范围来实现后三种方法。
\subsection{程序}
法一、二见附录A(一起实现)。法三见附录B。法四见附录C。
\subsection{结果与分析}
\subsubsection{例题}
输入增广矩阵
\begin{lstlisting}
-1 2 -2 -3
4 -3 1 5
2 1 -1 1
\end{lstlisting}
分别得到输出
\begin{lstlisting}
4	-3	1	5	
0	2.5	-1.5	-1.5	
0	0	-1	-1	
1 0 1 
time231

4	-3	1	5	
0	2.5	-1.5	-1.5	
0	0	-1	-1	
1 0 1 
time241

0	0	-1	-1	
4	-3	1	5	
0	2.5	-1.5	-1.5	
1 0 1 
time185

0	0	-1	-1	
4	0	0	4	
0	2.5	0	0	
1 0 1 
time269
\end{lstlisting}
方法二和方法一思路相同,不过由于多储存了一组数和多进行了一些运算,速度相应有所减缓,但是对不同特点的方程适应性更好。方法三在使用index数组储存行顺序的过程中很容易出现bug,逻辑比较复杂,但是速度有明显提高。法四运算量明显增大,不过有其独特作用。
\subsubsection{希尔伯特矩阵}
我们需要验证希尔伯特矩阵的求解是病态问题,所以取一组简单的解,如$$x=(1, \cdots , 1)^{T}$$
增广矩阵则需要取为
$$
\left( \begin{array} { c c c c c c} { 1 } & { \frac { 1 } { 2 } } & { \cdots } & {}& { \frac { 1 } { n } } & {\sum_{k=1}^{n} \frac{1}{k}} \\ { \frac { 1 } { 2 } } & { \ddots} & { } & { } &{\vdots}&{\sum_{k=2}^{n+1} \frac{1}{k}}\\ { \vdots } & { }  & {\frac{1}{i+j-1} }&{} &{}&{\sum_{k=i}^{i+n-1} \frac{1}{k}}\\ {}&{}&{}&{\ddots}&{}&{} \\{ \frac { 1 } { n } } & { \cdots} &{}&{}& { \frac { 1 } { 2 n } } &{\sum_{k=n}^{2n} \frac{1}{k}}\end{array} \right)
$$
赋值的代码如下
\begin{lstlisting}
for(int i=0;i<m;i++){
        for(int j=0;j<m;j++){
            a[i][j]=1.0/(j+i+1);
        }
    }
        for(int i=0;i<m;i++){
            a[i][m]=0;
            for(int j=0;j<m;j++){
                a[i][m]+=a[i][j];
            }
        }
\end{lstlisting}
运行程序,输出m取3到40时的解的二范数,与理论值对比
\begin{lstlisting}
实际解二范数	理论解二范数	差值
2		1.73205		0.267949
2.23607		2		0.236068
2.44949		2.23607		0.213422
2.64575		2.44949		0.196262
2.82843		2.64575		0.182676
3		2.82843		0.171573
3.16228		3		0.162278
3.31662		3.16228		0.154347
3.46422		3.31662		0.147595
3.61839		3.4641		0.154292
16.8017		3.60555		13.1962
5.94328		3.74166		2.20162
10.9974		3.87298		7.12445
11.0675		4		7.06752
41.7359		4.12311		37.6128
39.5347		4.24264		35.292
28.5012		4.3589		24.1423
83.0477		4.47214		78.5756
58.6187		4.58258		54.0361
92.9998		4.69042		88.3094
9922.75		4.79583		9917.95
286.163		4.89898		281.264
120.988		5		115.988
334.245		5.09902		329.146
257.786		5.19615		252.59
115.11		5.2915		109.818
293.836		5.38516		288.451
172.94		5.47723		167.463
242.542		5.56776		236.975
178.151		5.65685		172.495
102.009		5.74456		96.2645
148.099		5.83095		142.268
2295.96		5.91608		2290.04
190.215		6		184.215
281.252		6.08276		275.17
469.301		6.16441		463.137
288.553		6.245		282.308

\end{lstlisting}
换用方法二、三、四,差值的量级仍差不多。可以看出,对于迭代法,希尔伯特方程的解的确是病态问题。
\begin{appendices}
      \section{代码一}
      \begin{lstlisting}
#include <iostream>
#include <math.h>
#include <iomanip>
using namespace std;
int main() {
    int m=3,n=m+1;
    double a[m][n];
    for(int i=0;i<m;i++){
        for(int j=0;j<n;j++){
            cin>>a[i][j];
        }
    }
    clock_t start=clock();
    double temp2[n];
    double temp1[n];
    ///////////////////////方法二储存每行最大值
    double s[m];
    for(int i=0;i<m;i++){
        s[i]=a[i][0];
        for(int j=0;j<m;j++){
            if(a[i][j]>s[i]){
                s[i]=a[i][j];
            }
            
        }
    }
    
    for(int i=0;i<m;i++){
        ///////////////////////方法一找到最大位置
        /*
        int k=i;
        double max=abs(a[i][i]);
        for(int j=i;j<m;j++){
            if(fabs(a[j][i])>max){
                max=fabs(a[j][i]);
                k=j;
            }
        }
         */
        ///////////////////////方法二找到最大位置
        
         int k=i;
         double max=abs(a[i][i]/s[i]);
         for(int j=i;j<m;j++){
         if(fabs(a[j][i]/s[j])>max){
         max=fabs(a[j][i]/s[j]);
         k=j;
         }
         }
        
        ///////////////////////交换d

        for(int t=i;t<n;t++){
            temp2[t]=a[k][t];
        }
        for(int t=i;t<n;t++){
            a[k][t]=a[i][t];
        }
        for(int t=i;t<n;t++){
            a[i][t]=temp2[t];
        }
        ///////////////////////算系数
        for(int j=i+1;j<m;j++){
            temp1[j]=a[j][i]/temp2[i];
        }
        ///////////////////////相减
        for(int j=i+1;j<m;j++){
            for(int k=i;k<n;k++){
                a[j][k]-=temp1[j]*temp2[k];
            }
        }
    }
    ///////////////////////输出
    for(int q=0;q<m;q++){
        for(int w=0;w<n;w++){
            cout<<setprecision(2)<<a[q][w]<<"\t";
        }
        cout<<endl;
    }
    double results[m];
    for(int i=0;i<m;i++){
        results[i]=a[i][n-1];
    }
    results[m-1]=results[m-1]/a[m-1][m-1];
    for(int i=m-2;i>=0;i--){
        for(int j=m-1;j>i;j--){
            results[i]-=a[i][j]*results[j];
        }
        results[i]=results[i]/a[i][i];
    }
    clock_t end=clock();
    for(int i=0;i<m;i++){
        cout<<results[i]<<" ";
    }
    cout<<endl<<"time"<<(end-start)<<endl;
    return 0;
}


      \end{lstlisting}
 \section{代码二}
 \begin{lstlisting}
#include <iostream>
#include <math.h>
#include <iomanip>
using namespace std;
int main() {
    int m=3,n=m+1;
    double a[m][n];
    for(int i=0;i<m;i++){
        for(int j=0;j<n;j++){
            cin>>a[i][j];
        }
    }
    clock_t start=clock();
    double temp1[n];
    int index[m];
    for(int i=0;i<m;i++){
        index[i]=i;
    }
    ///////////////////////方法二储存每行最大值
    double s[m];
    for(int i=0;i<m;i++){
        s[i]=a[i][0];
        for(int j=0;j<m;j++){
            if(a[i][j]>s[i]){
                s[i]=a[i][j];
            }
            
        }
    }
    
    for(int i=0;i<m;i++){
        ///////////////////////方法一找到最大位置
        /*
         int k=i;
         double max=abs(a[i][i]);
         for(int j=i;j<m;j++){
         if(fabs(a[j][i])>max){
         max=fabs(a[j][i]);
         k=j;
         }
         }
         */
        ///////////////////////方法二找到最大位置
        int k=i;
        double max=abs(a[i][i]/s[i]);
        for(int j=i;j<m;j++){
            if(fabs(a[index[j]][i]/s[index[j]])>max){
                max=fabs(a[index[j]][i]/s[index[j]]);
                k=j;
            }
        }
        
        ///////////////////////交换行顺序数组的值
        
        int temp=index[k];
        index[k]=index[i];
        index[i]=temp;
        ///////////////////////算系数
        for(int j=i+1;j<m;j++){
            temp1[index[j]]=a[index[j]][i]/a[index[i]][i];
        }
        ///////////////////////相减
        for(int j=i+1;j<m;j++){
            for(int s=i;s<n;s++){
                a[index[j]][s]-=temp1[index[j]]*a[index[i]][s];
            }
        }
    }
    ///////////////////////输出
    for(int q=0;q<m;q++){
        for(int w=0;w<n;w++){
            cout<<setprecision(2)<<a[q][w]<<"\t";
        }
        cout<<endl;
    }
    double results[m];
    for(int i=0;i<m;i++){
        results[i]=a[i][n-1];
    }
    results[index[m-1]]=results[index[m-1]]/a[index[m-1]][m-1];
    for(int i=m-2;i>=0;i--){
        for(int j=m-1;j>i;j--){
            results[index[i]]-=a[index[i]][j]*results[index[j]];
        }
        results[index[i]]=results[index[i]]/a[index[i]][i];
    }
    clock_t end=clock();
    for(int i=0;i<m;i++){
        cout<<results[index[i]]<<" ";
    }
        cout<<endl<<"time"<<(end-start)<<endl;
    return 0;
}

\end{lstlisting}
\section{代码三}
\begin{lstlisting}
#include <iostream>
#include <math.h>
#include <iomanip>
using namespace std;
int main() {
    int m=3,n=m+1;
    double a[m][n];
    for(int i=0;i<m;i++){
        for(int j=0;j<n;j++){
            cin>>a[i][j];
        }
    }
    double temp1[n];
    int index[m];
    for(int i=0;i<m;i++){
        index[i]=i;
    }
    clock_t start=clock();
    ///////////////////////方法二储存每行最大值
    double s[m];
    for(int i=0;i<m;i++){
        s[i]=a[i][0];
        for(int j=0;j<m;j++){
            if(a[i][j]>s[i]){
                s[i]=a[i][j];
            }
            
        }
    }
    
    for(int i=0;i<m;i++){
        ///////////////////////方法一找到最大位置
        /*
         int k=i;
         double max=abs(a[i][i]);
         for(int j=0;j<m;j++){
         if(fabs(a[j][i])>max){
         max=fabs(a[j][i]);
         k=j;
         }
         }
         */
        ///////////////////////方法二找到最大位置
        int k=i;
        double max=abs(a[i][i]/s[i]);
        for(int j=0;j<m;j++){
            if(fabs(a[index[j]][i]/s[index[j]])>max){
                max=fabs(a[index[j]][i]/s[index[j]]);
                k=j;
            }
        }
        
        ///////////////////////交换行顺序数组的值
        
        int temp=index[k];
        index[k]=index[i];
        index[i]=temp;
        ///////////////////////算系数
        for(int j=0;j<m;j++){
            temp1[index[j]]=a[index[j]][i]/a[index[i]][i];
        }
        ///////////////////////相减
        for(int j=0;j<m;j++){
            if(j==i){continue;}
            for(int s=0;s<n;s++){
                a[index[j]][s]-=temp1[index[j]]*a[index[i]][s];
            }
        }
    }
    ///////////////////////输出
    for(int q=0;q<m;q++){
        for(int w=0;w<n;w++){
            cout<<setprecision(2)<<a[q][w]<<"\t";
        }
        cout<<endl;
    }
    double results[m];
    for(int i=0;i<m;i++){
        results[index[i]]=a[index[i]][n-1]/a[index[i]][i];
    }
    clock_t end=clock();
    for(int i=0;i<m;i++){
        cout<<results[index[i]]<<" ";
    }
    cout<<endl<<"time"<<(end-start)<<endl;
    return 0;
}

\end{lstlisting}
 \end{appendices}
\end{document}
