% !Mode:: "TeX:UTF-8"

\documentclass[a4paper,11pt,onecolumn,twoside]{article}
\usepackage{fancyhdr}
\usepackage{amsmath,amsfonts,amssymb,amsthm}
\usepackage{graphicx}
\usepackage{verbatim}
\usepackage{mathptmx}
\usepackage{booktabs}
\usepackage[labelfont=bf]{caption}
\usepackage{indentfirst}
\usepackage{caption}
\usepackage{setspace}
\usepackage{enumitem}
\usepackage{subfigure}
\usepackage{fontspec}
\usepackage{appendix}
\usepackage{listings}
\usepackage{xeCJK}
\newtheorem{lemma}{引理}[section]% Please change the following fonts if they are not available.
\setmainfont{Times New Roman}

\addtolength{\topmargin}{-54pt}
\setlength{\oddsidemargin}{-0.9cm}
\setlength{\evensidemargin}{\oddsidemargin}
\setlength{\textwidth}{17.00cm}
\setlength{\textheight}{24.50cm}

\renewcommand{\baselinestretch}{1.1}
\parindent 22pt

\title{\textbf{第二次上机实验报告}}
\author{
马宇恒
\\[2pt]
{\small \textit{匡亚明学院 171240510}}}
\date{}
\fancypagestyle{firststyle}
{
   \fancyhf{}
   \fancyhead[C]{数值计算实验报告}
   \fancyhead[R]{\thepage}
}

\pagestyle{fancy}
\fancyhf{}

\fancyhead[LE,RO]{\thepage}
\fancyhead[CE]{数值计算与实验}
\fancyhead[RE]{}
\fancyhead[CO]{}
\fancyhead[LO]{}

\setlist{nolistsep}
\captionsetup{font=small}

\newcommand{\supercite}[1]{\textsuperscript{\cite{#1}}}
\begin{document}
\maketitle
\thispagestyle{firststyle}
\setlength{\oddsidemargin}{ 1cm}
\setlength{\evensidemargin}{\oddsidemargin}
\setlength{\textwidth}{15.50cm}
\vspace{-.8cm}


\setcounter{page}{1}

\setlength{\oddsidemargin}{-.5cm}  % 3.17cm - 1 inch
\setlength{\evensidemargin}{\oddsidemargin}
\setlength{\textwidth}{17.00cm}

\section{实验目的}
\subsection{任务一}
编写并测试子程序,计算$y = x - \sin x$,使得有效位的丢失最多一位。\par 提示:根据定理知$$1 - \frac { \sin x } { x } \geq \frac { 1 } { 2 }$$\par 时,位的丢失可以限定在一位。
\subsection{任务二}计算$$y _ { n } = \int _ { 0 } ^ { 1 } x ^ { n } e ^ { x } d x \quad ( n \geq 0 )$$\par 提示:由分步积分法得$$y _ { n + 1 } = e - ( n + 1 ) y _ { n }$$
\subsection{任务三}考虑由$$\left\{ \begin{array} { l } { x _ { 0 } = 1 \quad x _ { 1 } = 1 / 3 } \\ { x _ { n + 1 } = \frac { 13 } { 3 } x _ { n } - \frac { 4 } { 3 } x _ { n - 1 } \quad ( n \geq 1 ) } \end{array} \right.$$\par 归纳定义的实数序列。算法稳定吗?将初值改为$x_{0}=1$和$x_{1}=4$数值稳定吗?

\section{任务一}
\subsection{算法}
根据提示,首先我们解出x的取值范围使得可以保证直接计算的精度被限制在一位。鉴于该计算只是粗略的限定不同计算方法的使用范围,无需太精确,所以我们使用二分法查找,并且记该区间为S。在S内,可以直接调用库math.h的函数sin来完成计算。而对于$S^{c}$,我们采用sin函数的泰勒展开式$$x-sin(x)=\sum_{k=1}^{\infty} \frac{(-1)^{k+1}x^{2k+1}}{(2k+1)!}$$值得一提的是,这个展开是在零点进行的,而$S^{c}$也是包含零点的一个小区间,所以我们有理由相信这一方法的可行性。\par
在$S^{c}$中进行计算时,采取以下两个策略
\begin{itemize}
\item 由于展开式的后项对于前项来说是相对小量,所以将所有项计算出以后从后向前相加,减小误差
\item 加项前后具有包含关系,为了避免重复计算,只在每次迭代中改变相加量(程序中的add变量)
\end{itemize}
\subsection{程序}
见附录A。
\subsection{结果与分析}
经过计算得到的边界值是1.89551,即$S^{c}=(-1.89551,1.89551)$。在这一范围内进行几组测试,输出x值、泰勒展开计算值、库函数计算值,结果如下
\begin{lstlisting}
x=0.1
0.000166 65873053664116546813966923679117826395668
0.000166 58335317185080093338456208584830164909363
x=0.2
0.00133 31372837327847436644789880233474832493812
0.00133 06692049387947029970291623612865805625916
x=0.3
0.0044 988161010260470026866919113217591075226665
0.0044 797933386604427141719497740268707275390625
x=0.4
0.010 66167874678513281805347645558867952786386
0.010 581657691349499739175143986358307301998138
x=0.5
0.020 820939916716237755300866751895227935165167
0.020 574461395796994622742204228416085243225098
x=0.6
0.035 969169283471318088185597616757149808108807
0.035 357526604964606420367090322542935609817505
x=0.7
0.05 7100049941734655478686732976711937226355076
0.05 578231276230893875833771744510158896446228
x=0.8
0.08 5225538186220348157995374549500411376357079
0.08 2643909100477253026895141374552622437477112
x=0.9
0.1 2130581480239144287480712591786868870258331
0.1 1667309037251671899326765924342907965183258
x=1
0.1 6638935108153077302439726281590992584824562
0.1 5852901519210349512434277130523696541786194
\end{lstlisting}
可以看到计算结果非常不理想,泰勒展开的计算值甚至远远大于直接使用库函数的误差(泰勒展开计算值与库函数结果相差远远大于真值与库函数结果的最大绝对误差),并且误差随着x绝对值的增大迅速增大。

\section{任务二}
\subsection{算法}
直接由迭代,输出每一个计算得到的$y_{n}$,此处我们输出前16个。
\subsection{程序}
见附录B。
\subsection{结果与分析}
\begin{lstlisting}
y0 1.71828
y1 1
y2 0.71828
y3 0.56344
y4 0.46452
y5 0.39568
y6 0.3442
y7 0.30888
y8 0.24724
y9 0.49312
y10 -2.21292
y11 27.0604
y12 -322.007
y13 4188.8
y14 -58640.5
y15 879611
\end{lstlisting}
误差不断增大,以至于数值最后不收敛于0,而原数列显然是收敛于0的。事实上,由$$\big{|} e_{y_{n+1}}\big{|} \leq \big{|}(n+1)e_{y_{n}}\big{|}+\big{|}e-E\big{|}$$显然数值不稳定。

\section{任务三}
\subsection{算法}
直接由迭代,输出每一个计算得到的$x_{n}$(先后输出两组初值),此处我们输出前16个。
\subsection{程序}
见附录C。
\subsection{结果与分析}
\begin{lstlisting}
x0 1	1
x1 0.3333333333333333148296163	4
x2 0.1111111111111110077986908	16
x3 0.03703703703703661170854033	64
x4 0.01234567901234397656329289	256
x5 0.004115226337441750363577153	1024
x6 0.001371742112455616446897722	4096
x7 0.0004572473707186708031906253	16384
x8 0.0001524157898400848936640989	65536
x9 5.080526168214012818066458e-05	262144
x10 1.693508083582736728554645e-05	1048576
x11 5.645001379065088741536754e-06	4194304
x12 1.881564861512228377688903e-06	16777216
x13 6.267792277995379106834972e-07	67108864
x14 2.07290171781693100553357e-07	268435456
x15 6.255177398795287932995769e-08	1073741824
\end{lstlisting}
从结果中并不能看出误差的收敛性。然而,我们可以通过“特征根法”直接求得数列的通项,两组初值的通项分别为$$x_{n}=\frac{1}{3^{n}}$$$$x_{n}=4^{n}$$所以我们将误差进行输出(代码中的注释行),得到以下结果
\begin{lstlisting}
x0 0	0
x1 0	0
x2 -9.714451465470119728706777e-17	0
x3 -4.232725281383409310365096e-16	0
x4 -1.701763729933247759618098e-15	0
x5 -6.80965700494695624911401e-15	0
x6 -2.723667645587735108847482e-14	0
x7 -1.089469226639439014547861e-13	0
x8 -4.357876906557756058191444e-13	0
x9 -1.743150728741784533104564e-12	0
x10 -6.972602918355269921435458e-12	0
x11 -2.789041167342107968574183e-11	0
x12 -1.115616466924137693220859e-10	0
x13 -4.462465867690198025779028e-10	0
x14 -1.784986347075867452074797e-09	0
x15 -7.139945388303443338519588e-09	0
\end{lstlisting}
这里便可以清晰的看到,第一组初值是数值不稳定的,而第二组初值是数值稳定的,且误差一直是0。事实上,$\frac{1}{3}$无法被计算机浮点数系统完全表示是造成第一组初值有误差的根本原因。从$x_{1}$开始便出现了误差,而由$$ \big{|}e_{x _ { n + 1 }} \big{|}\leq \big{|}\frac { 13 } { 3 } e_{x _ { n }} - \frac { 4 } { 3 } e_{x _ { n - 1 }}\big{|}, \quad e_{x_{1}}>e_{x_{0}}$$,我们可以看出误差数列应当是指数速度单增的。而由于$2^{n}$都可以被计算机完全表示,运算过程中也没有出现不能完全表示的浮点数,所以第二组初值的数列是数值稳定的。
\begin{appendices}
      \section{代码一}
      \begin{lstlisting}
#include<iostream>
#include<math.h>
#include <iomanip>
using namespace std;
double Abs(double x){
    if(x>=0){return x;}
    else{return -x;}}
int sgn(float x){
    if(x>0){return 1;}
    else if(x==0){return 0;}
    else{return -1;}
}
double f(float x){
    return x-sin(x);
}

int main(){
    float a=1,b=2,temp=1.5,epsilon=0.001;
    while(a-b>epsilon||a-b<-epsilon){
        temp=(a+b)/2;
        if(sgn(sin(temp)/temp-0.5)!=sgn(sin(b)/b-0.5)){a=temp;}
        else{b=temp;}
    }
    float x;
    cin>>x;
    if(x<=a&&x>=-a){
        double sum=0,add=-x*x*x/6;
        int count=2;
        int i=0;
        while(Abs(add)>pow(10,-20))
        {add=-add*x*x/(2*count*(2*count+1));i++;count++;}
        //while(add!=0){add=-add*x*x/(2*count*(2*count+1));i++;count++;}
        double adds[i];
        add=x*x*x/6;
        for(int j=0;j<i;j++){adds[j]=add;
            //cout<<setprecision(45)<<add<<endl;
            add=-add*x*x/(2*count*(2*count+1));}
        for(int j=i-1;j>=0;j--){sum=sum+adds[j];}
        cout<<setprecision(44)<<sum<<endl;

    }
    else{cout<<f(x);}
    return 0;
}
      \end{lstlisting}
 \section{代码二}
 \begin{lstlisting}
#include <iostream>
#define E 2.71828
using namespace std;
int main() {
    double temp,I=E-1,N=50,epsilon=0.0001;
    for(int i=0;i<N;i++){
        cout<<"y"<<i<<" "<<I<<endl;
        I=E-(i+1)*I;
    }
    return 0;
}
\end{lstlisting}

 \section{代码三}
 \begin{lstlisting}
#include <iostream>
#include <math.h>
#include <iomanip>
using namespace std;
int main() {
    int N=30;
    double a[N],b[N];
    a[0]=1;a[1]=1.0/3;
    b[0]=1;b[1]=4;
    for(int i=2;i<N;i++){
        a[i]=(13*a[i-1]-4*a[i-2])/3;
        b[i]=(13*b[i-1]-4*b[i-2])/3;
    }
    for(int i=0;i<N;i++)
    {cout<<setprecision(25)<<"x"<<i<<" "<<a[i]<<"\t"<<b[i]<<endl;}
    //for(int i=0;i<N;i++)
    //{cout<<setprecision(25)<<"x"<<i
    <<" "<<a[i]-pow(3,-i)<<"\t"<<b[i]-pow(4,i)<<endl;}
    return 0;
}
\end{lstlisting}
\end{appendices}
\end{document}
