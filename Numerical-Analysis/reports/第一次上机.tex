% !Mode:: "TeX:UTF-8"

\documentclass[a4paper,11pt,onecolumn,twoside]{article}
\usepackage{fancyhdr}
\usepackage{amsmath,amsfonts,amssymb,amsthm}
\usepackage{graphicx}
\usepackage{verbatim}
\usepackage{mathptmx}
\usepackage{appendix}
\usepackage{booktabs}
\usepackage[labelfont=bf]{caption}
\usepackage{indentfirst}
\usepackage{caption}
\usepackage{setspace}
\usepackage{enumitem}
\usepackage{subfigure}
\usepackage{fontspec}
\usepackage{listings}
\usepackage{xeCJK}
\newtheorem{lemma}{引理}[section]% Please change the following fonts if they are not available.
\setmainfont{Times New Roman}

\addtolength{\topmargin}{-54pt}
\setlength{\oddsidemargin}{-0.9cm}
\setlength{\evensidemargin}{\oddsidemargin}
\setlength{\textwidth}{17.00cm}
\setlength{\textheight}{24.50cm}

\renewcommand{\baselinestretch}{1.1}
\parindent 22pt

\title{\textbf{第一次上机实验报告}}
\author{
马宇恒
\\[2pt]
{\small \textit{匡亚明学院 171240510}}}
\date{}
\fancypagestyle{firststyle}
{
   \fancyhf{}
   \fancyhead[C]{数值计算实验报告}
   \fancyhead[R]{\thepage}
}

\pagestyle{fancy}
\fancyhf{}

\fancyhead[LE,RO]{\thepage}
\fancyhead[CE]{数值计算与实验}
\fancyhead[RE]{}
\fancyhead[CO]{}
\fancyhead[LO]{}

\setlist{nolistsep}
\captionsetup{font=small}

\newcommand{\supercite}[1]{\textsuperscript{\cite{#1}}}
\begin{document}
\maketitle
\thispagestyle{firststyle}
\setlength{\oddsidemargin}{ 1cm}
\setlength{\evensidemargin}{\oddsidemargin}
\setlength{\textwidth}{15.50cm}
\vspace{-.8cm}


\setcounter{page}{1}

\setlength{\oddsidemargin}{-.5cm}  % 3.17cm - 1 inch
\setlength{\evensidemargin}{\oddsidemargin}
\setlength{\textwidth}{17.00cm}

\section{实验目的}
编程序找出所使用计算机的机器精度、下溢值和上溢值。

\section{理论知识}
\begin{itemize}
    \item 机器精度:用$\epsilon_{mach}$来指代,也叫单位舍入误差。它说明了使用特定p, t, L, U的计算机系统表示一个非零实数x的最大可能相对误差,即:$$\left| \frac { f l ( x ) - x } { x } \right| \leq \varepsilon _ { m a c h }$$
    \item 下溢值:特定p, t, L, U的浮点数系统能表示的最小值。
    \item 上溢值:特定p, t, L, U的浮点数系统能表示的最大值。
\end{itemize}

\section{算法}
\subsection{求机器精度}
机器精度是由浮点系统尾数域的位数决定的,尾数域为00$\cdots$01,在多数情况下有$$\varepsilon_{mach}=2^{1-t}$$为了便于计算,我们将x设定为1,y设定为$1+\frac{1}{2^{k}}$,不断比较两者的大小。当两者即将一样大时,令fl(x)=y带入计算即得结果。
\subsection{求下溢值}
下溢值是由指数域的位数决定的,其指数域为00$\cdots$01,在多数情况下有$$UFL=2^{-(L+1)}$$通过不断的比较$\frac{1}{2^{k}}$和0的大小找到。
\subsection{求上溢值}
上溢值是由指数域的位数决定的,其指数域为1$\cdots$11,在多数情况下有$$OFL=2^{U}(1-2^{-t})$$首先找到超过上溢值之前的2的次幂,即$2^{U-1}$,亦即指数域为0.10$\cdots$0的情况。截断方法是通过比较一个数的2倍与4倍是否相等,若达到$2^{U-1}$,则应当相等。而后不断与$2^{U-1-k}, k=1,2\cdots$相加,直至$\sum_{U-1-k}^{U-1}2^{i}$超限。
\section{程序}
见附录A。
\section{结果与分析}
\subsection{结果}
运行所示代码,输出结果如下。
\begin{lstlisting}
单精度上溢3.40282e+38
7f7fffff
单精度下溢1.4013e-45
1
单精度机器精度5.96046e-08
33800000
双精度上溢1.79769e+308
1.79769e+308
双精度下溢4.94066e-324
4.94066e-324
双精度机器精度1.11022e-16
2.22045e-16
Program ended with exit code: 0
\end{lstlisting}
\subsection{分析}
在单精度的部分,采用了输出十六进制表示的方法进行验证。\par  单精度上溢值的十六进制表示是7f7fffff,转换成二进制并把位数补齐是$$0\overbrace{11111110}^{8}\underbrace{11111111111111111111111}_{23}$$还原左移规范化后即$2^{128}\times(1-2^{-24})$,正是我们所要求取的值。然而其二进制表示与想象的稍有不同,为何不是全部位皆为1时是最大值?通过查找资料得知,其指数位是254,而指数位为255时是为NaN保留的。所以求出的结果正确。\par 单精度下溢值十六进制是1,转换成二进制是$$00000000000000000000000000000001$$结果是$1\times 2^{-23}\times 2^{-126}=2^{-149}$。此处没有使用左移规范化的原因经过查阅资料,是因为在处理较小的量时,计算机会进行非规范化操作,获取更多的小数位。所以,求出的结果是正确的。\par
单精度机器精度的十六进制表示是33800000,转换成二进制是$$0\underbrace{01100111}_{8}00000000000000000000000$$\par 即$1.0\times 2^{64+32+4+2+1-127}=2^{-24}$。这与t=23的结论相符,所以,求出的结果是正确的。\par
双精度的部分采用了库函数直接输出数值的方法进行验证。从输出可以看出,上溢值和下溢值求出的结果均正确。而我们所采用的输出机器精度的库函数是“运行编译程序的计算机所能识别的最小非零浮点数”,所以其结果自然是机器精度的二倍。
\section{结论}
求得的结果最终为
\begin{itemize}
\item 单精度上溢值$3.40282\times10^{38}$
\item 单精度下溢值$1.4013\times10^{-45}$
\item 单精度机器精度$5 . 9 6 0 4 6\times10^{-8}$
\item 双精度上溢值$1.79769\times10^{308}$
\item 双精度下溢值$4.94066\times10^{-324}$
\item 双精度机器精度$1 . 1 1 0 2 2\times10^{-16}$
\end{itemize}
					

\begin{appendices}
      \section{代码}
      \begin{lstlisting}
#include <iostream>
#include <math.h>
#include <string.h>
using namespace std;
int main() {
    float i=0,supof=1,infof=1,mepre=2,temp=0,me=0;
    while(supof!=2*supof){
        temp=supof;
        supof=supof*2;
        i++;
    }
    supof=temp;
    while((temp/2+supof)*2!=temp/2+supof){
        supof=temp/2+supof;
        temp=temp/2;
    }
    while(infof/2>0){
        infof=infof/2;
        i++;
    }
    while((mepre+1)/2!=1){
        mepre=(mepre+1)/2;
    }
    me=(mepre-1)/2;
    cout<<"单精度"<<"上溢"<<supof<<endl;
    printf("%x\n", *(int *)&supof);
    cout<<"单精度"<<"下溢"<<infof<<endl;
    printf("%x\n", *(int *)&infof);
    //char a[16]=(char *)&infof;
    //char *a=(char *)&infof;
    //for(int i=0;i<16;i++){cout<<a[i];}
    cout<<"单精度"<<"机器精度"<<me<<endl;
    printf("%x\n", *(int *)&me);
    //ofstream outfile("out1.bin",ios::out|ios::binary);
    //outfile.write((char*)&b,sizeof(float));
    //
    double i2=0,supof2=1,infof2=1,mepre2=2,me2=0,temp2=0;
    while(supof2!=2*supof2){
        temp2=supof2;
        supof2=supof2*2;
        i2++;
    }
    supof2=temp2;
    while((temp2/2+supof2)*2!=temp2/2+supof2){
        supof2=temp2/2+supof2;
        temp2=temp2/2;
    }
    while(infof2/2>0){
        infof2=infof2/2;
        i2++;
    }
    while((mepre2+1)/2!=1){
        mepre2=(mepre2+1)/2;
    }
    me2=(mepre2-1)/2;
    cout<<"双精度"<<"上溢"<<supof2<<endl;
    双上溢值: cout<<std::numeric_limits<double>::max()<<endl;
    cout<<"双精度"<<"下溢"<<infof2<<endl;
    双规约下溢值: cout<<std::numeric_limits<double>::denorm_min()<<endl;
    cout<<"双精度"<<"机器精度"<<me2<<endl;
    双机器精度: cout<<std::numeric_limits<double>::epsilon()<<endl;
    return 0;
}


      \end{lstlisting}
       \end{appendices}
\end{document}
