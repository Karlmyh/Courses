% !Mode:: "TeX:UTF-8"

\documentclass[a4paper,11pt,onecolumn,twoside]{article}
\usepackage{fancyhdr}
\usepackage{amsmath,amsfonts,amssymb,amsthm}
\usepackage{graphicx}
\usepackage{verbatim}
\usepackage{mathptmx}
\usepackage{booktabs}
\usepackage[labelfont=bf]{caption}
\usepackage{indentfirst}
\usepackage{caption}
\usepackage{setspace}
\usepackage{enumitem}
\usepackage{subfigure}
\usepackage{fontspec}
\usepackage{appendix}
\usepackage{listings}
\usepackage{xeCJK}
\newtheorem{lemma}{引理}[section]% Please change the following fonts if they are not available.
\setmainfont{Times New Roman}

\addtolength{\topmargin}{-54pt}
\setlength{\oddsidemargin}{-0.9cm}
\setlength{\evensidemargin}{\oddsidemargin}
\setlength{\textwidth}{17.00cm}
\setlength{\textheight}{24.50cm}

\renewcommand{\baselinestretch}{1.1}
\parindent 22pt

\title{\textbf{第七次上机实验报告}}
\author{
马宇恒
\\[2pt]
{\small \textit{匡亚明学院 171240510}}}
\date{}
\fancypagestyle{firststyle}
{
   \fancyhf{}
   \fancyhead[C]{数值计算实验报告}
   \fancyhead[R]{\thepage}
}

\pagestyle{fancy}
\fancyhf{}

\fancyhead[LE,RO]{\thepage}
\fancyhead[CE]{数值计算与实验}
\fancyhead[RE]{}
\fancyhead[CO]{}
\fancyhead[LO]{}

\setlist{nolistsep}
\captionsetup{font=small}

\newcommand{\supercite}[1]{\textsuperscript{\cite{#1}}}
\begin{document}
\maketitle
\thispagestyle{firststyle}
\setlength{\oddsidemargin}{ 1cm}
\setlength{\evensidemargin}{\oddsidemargin}
\setlength{\textwidth}{15.50cm}
\vspace{-.8cm}


\setcounter{page}{1}

\setlength{\oddsidemargin}{-.5cm}  % 3.17cm - 1 inch
\setlength{\evensidemargin}{\oddsidemargin}
\setlength{\textwidth}{17.00cm}

\section{实验目的}
\subsection{求$\pi$}
数学上已经证明$$\int _ { 0 } ^ { 1 } \frac { 4 } { 1 + x ^ { 2 } } \mathrm { d } x = \pi.$$成立,所以可以通过数值积分来求圆周率的近似值。
\begin{itemize}
\item 分别使用复合梯形,复合 Simpson 求积公式计算$\pi$的近似值。选择不同的 h , 对每种求积公式,试将误差刻画成h的函数,并比较两方法的精度。是否存在某个h值,当低于这个值之后再继续减小h的值,计算不再有所改进? 为什么?
\item 实现Romberg求积方法,并重复上面的计算。
\item 实现自适应的积分方法,并重复上面的计算。
\end{itemize}
\subsection{计算积分}
Planck关于黑体辐射的理论推出积分$$\int _ { 0 } ^ { \infty } \frac { x ^ { 3 } } { e ^ { x } - 1 } \mathrm { d } x$$用所掌握的所有数值积分方法计算这个积分,比较不同方法的计算效率和精度。
\section{任务一}
将四种方法写成函数的形式。其中复合梯形和复合Simpson输入的是步长,而自适应和Romberg输入的是误差容限。自适应法采取的是调用自身的迭代法。
\subsection{程序}
\begin{lstlisting}
#include <iostream>
#include <iomanip>
#include <math.h>
using namespace std;
#define PI 3.141592653589;
double f(double x){
    return 4.0/(1+x*x);
}
double T(double h,double a=0,double b=1){
    double I=(f(a)+f(b))/2;
    double x=a;
    while(x+h<b){
        x+=h;
        I=I+f(x);
    }
    return I*h;
}
double S(double h,double a=0,double b=1){
    double I=(f(a)+f(b))/3;
    double x=a;
    int n=1;
    while(x+h<b){
        x+=h;
        I=I+2*f(x)/3+2*n*f(x)/3;
        n=1-n;
    }
    return h*I;
}

double half(double epsilon,double a=0,double b=1){
    double I=(f(a)+f(b))/2;
    double temp=0;
    double h=1;
    while(abs(I-temp)>epsilon){
        temp=I;
        I=0.5*I;
        double step=0.5*h;
        while(step<b){
            I=I+0.5*h*f(step);
            step+=h;
        }
        h=0.5*h;
    }
    return I;
}
double romberg(double epsilon,double a=0,double b=1){
    double count[20];//changable
    int n=0;
    double I=(f(a)+f(b))/2;
    count[0]=I;n++;
    double temp=0;
    double h=1;
    while(abs(I-temp)>epsilon){
        temp=I;
        I=0.5*I;
        double step=0.5*h;
        while(step<b){
            I=I+0.5*h*f(step);
            step+=h;
        }
        count[n]=I;n++;
        h=0.5*h;
    }
    n=n-1;
    double T[n][n];
    T[0][0]=(f(a)+f(b))/2;
    for(int i=0;i<n;i++){
        T[i][0]=count[i];
    }
    for(int j=1;j<n;j++){
        for(int i=j;i<n;i++){
            T[i][j]=(pow(4,j)*T[i][j-1]-T[i-1][j-1])/(pow(4,j)-1);
        }
    }
    return T[n-1][n-1];
}
double self(double a,double b,double epsilon){
    if(abs(S((b-a)/2,a,b)-S((b-a)/4,(a+b)/2,b)-S((b-a)/4,a,(a+b)/2))<epsilon){
        return S((b-a)/2,a,b);
    }
    else return self((a+b)/2,b,epsilon)+self(a,(a+b)/2,epsilon);
}
double gausslegendre(double h,double a=0,double b=1){
    double I=0;
    double x=a-h/2;
    while(x+h<b){
        x+=h;
        I=I+h/2*(f(sqrt(3)*h/6+x)+f(-sqrt(3)*h/6+x));
    }
    return I;
}
int main() {
    cout<<setprecision(12)<<T(0.000001)<<endl;
    cout<<S(0.0000005)<<endl;
    cout<<half(0.0001)<<endl;
    cout<<self(0,1,0.000001)<<endl;
    cout<<romberg(0.0001)<<endl;
    cout<<gausslegendre(0.0001)<<endl;
    return 0;
}

\end{lstlisting}
\subsection{结果与分析}
对某个h,如果最大误差$\max_{x} R_h(x)$已经小于舍入误差,那么计算精度不会再改进。

\section{任务二}
首先分析可行的方法。\par
第一,由$$\int_0^{\infty}\frac{x^3}{e^x-1}dx=\frac{\pi ^4}{15},$$之前所有可以计算$\pi$的方法均可以在此处使用,本次报告掠过不提。精确值约为6.49393940226682。\par
由于函数在x趋于无穷时趋于0,所以区间截断法可以使用。在截断后的区间上,使用本章节介绍的方法求解并且计时。\par
通过变量代换,可以把无穷积分转换为有穷积分,现在尝试以下三种代换。
$$\int_0^{\infty}\frac{x^3}{e^x-1}dx\overset{u=e^{-x}}{=}\int^1_{0}\frac{ln^3u}{u-1}du$$
$$\int_0^{\infty}\frac{x^3}{e^x-1}dx\overset{u=arctanx}{=}\int^{\frac{\pi}{2}}_{0}\frac{1}{cos^2u}\frac{tan^3u}{e^{tanu}-1}du$$
$$\int_0^{\infty}\frac{x^3}{e^x-1}dx=\int_0^{1}\frac{x^3}{e^x-1}dx+\int_1^{\infty}\frac{x^3}{e^x-1}dx\overset{x=\frac{1}{u}}{=}\int^1_{0}\frac{u^3}{e^u-1}+\frac{1}{u^5(e^{\frac{1}{u}}-1)}du$$
\subsection{计算截断区间}
由$$\int_b^{\infty}\frac{x^3}{e^x-1}dx<\int_b^{\infty}\frac{2e^{x/2}}{e^x}dx=4e^{-\frac{b}{2}}.$$
我们有积分区间$$(0,  ln \frac{16}{\epsilon^2}).$$
在这个区间上,我们用cut函数输出指定精度和种类的积分值与花费的时间。
\subsection{程序}
\begin{lstlisting}
#include <iostream>
#include <iomanip>
#include <math.h>
using namespace std;
#define PI 3.141592653589;
double f(double x){
    return pow(x,3)/(exp(x)-1);
}
double T(double h,double a=0,double b=1){
    double I=(f(a)+f(b))/2;
    double x=a;
    while(x+h<b){
        x+=h;
        I=I+f(x);
    }
    return I*h;
}
double S(double h,double a=0,double b=1){
    double I=(f(a)+f(b))/3;
    double x=a;
    int n=1;
    while(x+h<b){
        x+=h;
        I=I+2*f(x)/3+2*n*f(x)/3;
        n=1-n;
    }
    return h*I;
}

double half(double epsilon,double a=0,double b=1){
    double I=(f(a)+f(b))/2;
    double temp=-5;
    double h=b-a;
    while(abs(I-temp)>epsilon){
        temp=I;
        I=0.5*I;
        double step=0.5*h;
        while(step<b){
            I=I+0.5*h*f(step);
            step+=h;
        }
        h=0.5*h;
    }
    return I;
}
double romberg(double epsilon,double a=0,double b=1){
    double count[100];//changable
    int n=0;
    double I=(f(a)+f(b))/2;
    count[0]=I;n++;
    double temp=-5;
    double h=b-a;
    while(abs(I-temp)>epsilon){
        temp=I;
        I=0.5*I;
        double step=0.5*h;
        while(step<b){
            I=I+0.5*h*f(step);
            step+=h;
        }
        count[n]=I;n++;
        h=0.5*h;
    }
    n=n-1;
    double T[n][n];
    T[0][0]=(f(a)+f(b))/2;
    for(int i=0;i<n;i++){
        T[i][0]=count[i];
    }
    for(int j=1;j<n;j++){
        for(int i=j;i<n;i++){
            T[i][j]=(pow(4,j)*T[i][j-1]-T[i-1][j-1])/(pow(4,j)-1);
        }
    }
    return T[n-1][n-1];
}
double self(double a,double b,double epsilon){
    if(abs(S((b-a)/2,a,b)-S((b-a)/4,(a+b)/2,b)-S((b-a)/4,a,(a+b)/2))<epsilon){
        return S((b-a)/2,a,b);
    }
    else return self((a+b)/2,b,epsilon)+self(a,(a+b)/2,epsilon);
}
double gausslegendre(double h,double a=0,double b=1){
    double I=0;
    double x=a-h/2;
    while(x+h<b){
        x+=h;
        I=I+h/2*(f(sqrt(3)*h/6+x)+f(-sqrt(3)*h/6+x));
    }
    return I;
}
double cut(double epsilon,int n){
    double I=0;
    double a=epsilon/100,b=log(16/epsilon/epsilon);
    clock_t start=clock();
    if(n==1){I=T(1,a,b);cout<<"T";}
    if(n==2){I=S(1,a,b);cout<<"Simpson";}
    if(n==3){I=half(epsilon,a,b);cout<<"half";}
    if(n==4){I=romberg(epsilon,a,b);cout<<"romberg";}
    if(n==5){I=self(a,b,epsilon);cout<<"self";}
    if(n==6){I=gausslegendre(1,a,b);cout<<"gausslegendre";}
    clock_t end=clock();
    cout<<endl<<"time"<<(end-start)<<endl;
    return I;
}
int main() {
    for(int i=1;i<7;i++){
        cout<<setprecision(15)<<cut(0.0001,i)<<endl<<endl;}
    return 0;
}

\end{lstlisting}
\subsection{结果与分析}
\begin{lstlisting}
T
time51
6.48977043637157

Simpson
time11
6.5106022369413

half
time14
6.49392891509266

romberg
time16
6.49393339456984

self
time58
6.49400443457754

gausslegendre
time7
6.49323683380132

\end{lstlisting}
由于输入输出标准不同(有的输入误差容限,有的输入步长)。我们只能相对的比较各自的误差和时间。复合梯形积分时间超长,精度还不咋样,非常差劲。而辛普森积分用第二短的时间拿下了最大的误差。不过我们也应该注意到,由于这两种方法选取积分值点的僵化思维,使得在函数值比较大的点对于积分值影响较大。Legendre算法用了最短的时间,达到了相对优秀的精度。并且在我们减小步长之后精度迅速提高,是这三者里最优秀的。对于三个输入误差容限的函数,自适应,区间分半和Romberg,我们减小误差容限之后看出,自适应的运算时间较长,这是由其调用自身的特点决定的。并且由于每次判断的标准设为了Epsilon,最终的误差累计导致了精度降低。对区间分半和Romberg来说,由于加速的原因,减小误差容限之后Romberg的精度明显高于区间分半,而时间几乎没有增加。所以Romberg算法是较为优秀的选择,缺点就是代码太长,不可徒手写出。
\subsection{积分变量代换的做法}
对三种变换分别使用cut函数。代码中只修改f的取值和积分区间,所以此处不附。
\subsection{结果与分析}
第一种变换,只有勒让德算法和自适应算法可以快速输出符合精确要求的值,S和T误差过大,而其他两种则循环时间过长。从此处可以看出勒让德算法的精度相对高,而自适应算法应对导数变化较大的情况有很好的提高,避免了大量的无用功。
第二种变换,情形非常糟糕,S和T的返回值已经失真,只有勒让德还保持了基本的素养,返回了相当精度的值。另外三种算法则花费极其漫长的时间,并且返回值也有一定失真。
而到了第三种变换,输出的精度有了质的提高,基本可以达到我们所要的任意精度。分析原因是我们把积分转化为了有限区间上的有界积分,这是最简单的数值积分情况,计算效果必然好。但同时,计算时间也飞速增长,相比于之前的不超过2位,这里的运算时间往往4~5位数起步。分析原因有函数变得复杂调用耗时长,函数变化相对剧烈、取值点变多。
\end{document}
