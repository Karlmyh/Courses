% !Mode:: "TeX:UTF-8"

\documentclass[a4paper,11pt,onecolumn,twoside]{article}
\usepackage{fancyhdr}
\usepackage{amsmath,amsfonts,amssymb,amsthm}
\usepackage{graphicx}
\usepackage{verbatim}
\usepackage{mathptmx}
\usepackage{booktabs}
\usepackage[labelfont=bf]{caption}
\usepackage{indentfirst}
\usepackage{caption}
\usepackage{setspace}
\usepackage{enumitem}
\usepackage{subfigure}
\usepackage{fontspec}
\usepackage{appendix}
\usepackage{listings}
\usepackage{xeCJK}
\newtheorem{lemma}{引理}[section]% Please change the following fonts if they are not available.
\setmainfont{Times New Roman}

\addtolength{\topmargin}{-54pt}
\setlength{\oddsidemargin}{-0.9cm}
\setlength{\evensidemargin}{\oddsidemargin}
\setlength{\textwidth}{17.00cm}
\setlength{\textheight}{24.50cm}

\renewcommand{\baselinestretch}{1.1}
\parindent 22pt

\title{\textbf{第三次上机实验报告}}
\author{
马宇恒
\\[2pt]
{\small \textit{匡亚明学院 171240510}}}
\date{}
\fancypagestyle{firststyle}
{
   \fancyhf{}
   \fancyhead[C]{数值计算实验报告}
   \fancyhead[R]{\thepage}
}

\pagestyle{fancy}
\fancyhf{}

\fancyhead[LE,RO]{\thepage}
\fancyhead[CE]{数值计算与实验}
\fancyhead[RE]{}
\fancyhead[CO]{}
\fancyhead[LO]{}

\setlist{nolistsep}
\captionsetup{font=small}

\newcommand{\supercite}[1]{\textsuperscript{\cite{#1}}}
\begin{document}
\maketitle
\thispagestyle{firststyle}
\setlength{\oddsidemargin}{ 1cm}
\setlength{\evensidemargin}{\oddsidemargin}
\setlength{\textwidth}{15.50cm}
\vspace{-.8cm}


\setcounter{page}{1}

\setlength{\oddsidemargin}{-.5cm}  % 3.17cm - 1 inch
\setlength{\evensidemargin}{\oddsidemargin}
\setlength{\textwidth}{17.00cm}

\section{实验目的}
利用所学的格式求方程$$e^{-x}-sinx=0$$的根,并比较各种格式的优缺点。给出新的迭代格式(至少一种),求解上述方程,并分析所提格式的收敛性。


\section{实验内容}
\subsection{算法}
学过的格式总结为二分法,不动点迭代法,史蒂芬孙迭代法,牛顿迭代法,弦割法。于是我们分别编写程序输出其$(0, +\infty)$内第一个解(约为0.6)。
\subsection{程序}
见附录A。
\subsection{结果与分析}
各方法求解结果如下
\begin{lstlisting}
迭代精度             0.000001
二分法               0.5885326624
史蒂芬孙迭代法一     0.588532744
不动点法一           0.5885328714
史蒂芬孙迭代法二     0.588532744
不动点法二           0.5885330544
牛顿迭代法           0.5885327428
弦割法               0.5885327443
马勒法		     0.588529
\end{lstlisting}
\par 可以看到,0.6附近的收敛效果很好。接下来我们分别对每个函数分别输出迭代过程值。输出的结果数据处理的R语言代码见附录B。
\begin{figure}[htbp]
  \centering
  \includegraphics[width=0.7\textwidth]{result.png}
  \caption{各方法结果} \label{result}
\end{figure}

\par 首先,在需要考虑的区间较大的情况下,二分法的收敛速度是几种方法中最慢的。其次,史蒂芬孙加速和不动点迭代都需要寻找函数使得x=g(x)。本题中共有三种可选择的迭代方法
$$e^{-x}-sinx+x=x$$$$-ln sin x=x$$$$arcsin e^{-x}=x$$第一种和第三种在真解的一个区间内满足应用条件,可以使用0.6作为初值迭代,而第二种会出现对数里包含负数的情况,不能使用。将两个可用的g(x)分别使用史蒂芬孙加速和不动点迭代,发现两个g(x)相差不多,但是史蒂芬孙加速明显快于不动点迭代,不愧是史蒂芬孙加速!与此同时,牛顿迭代和史蒂芬孙加速具有差不多的速度,弦割迭代略慢于牛顿迭代。这些结论都符合课上给出的推导。
\subsection{进一步探索}
接下来我们在一个范围内输出各方法的迭代结果。
\begin{lstlisting}
x	二分法	史一	不动点一 史二	不动点二 牛顿	弦割     马勒
1	0.58853	0.58853	0.58853	0.58853	0.58853	0.58853	0.58853	0.59018
2	0	12.566	0.58853	0.58853	0.58853	25.133	9.4247	2.9787
3	3.0964	3.0964	0.58853	0.58853	0.58853	3.0964	3.0964	3.0964
4	3.0964	0.58853	6.285	0.58853	0.58853	3.0964	3.0964	3.1744
5	6.285	nan	6.285	0.58853	0.58853	9.4247	9.4247	6.1523
6	6.285	6.285	6.285	0.58853	0.58853	6.285	6.285	6.2819
7	6.285	6.285	6.285	0.58853	0.58853	6.285	6.285	6.3269
8	9.4247	nan	6.285	0.58853	0.58853	15.708	15.708	9.2753
9	9.4247	9.4247	6.285	0.58853	0.58853	9.4247	9.4247	9.4142
10	9.4247	9.4247	12.566	0.58853	0.58853	9.4247	9.4247	9.449
11	0	12.566	12.566	0.58853	0.58853	235.62	219.91	12.41
12	12.566	12.566	12.566	0.58853	0.58853	12.566	12.566	12.543
13	12.566	nan	12.566	0.58853	0.58853	12.566	12.566	12.578
14	12.566	nan	12.566	0.58853	0.58853	6.285	12.566	12.716
15	15.708	15.708	12.566	0.58853	0.58853	15.708	15.708	15.667
16	15.708	15.708	18.85	0.58853	0.58853	15.708	15.708	15.711
17	15.708	21.991	18.85	0.58853	0.58853	12.566	15.708	15.843
18	18.85	18.85	18.85	0.58853	0.58853	18.85	18.85	18.787
19	18.85	nan	18.85	0.58853	0.58853	18.85	18.85	18.85
20	18.85	18.85	18.85	0.58853	0.58853	18.85	18.85	18.963
21	21.991	25.133	18.85	0.58853	0.58853	21.991	21.991	21.905
22	21.991	21.991	25.133	0.58853	0.58853	21.991	21.991	21.991
23	21.991	12.566	25.133	0.58853	0.58853	21.991	21.991	22.081
24	25.133	25.133	25.133	0.58853	0.58853	25.133	25.133	25.022
25	25.133	nan	25.133	0.58853	0.58853	25.133	25.133	25.133
26	25.133	25.133	25.133	0.58853	0.58853	25.133	25.133	25.198
27	28.274	21.991	25.133	0.58853	0.58853	31.416	31.416	28.142
28	28.274	nan	25.133	0.58853	0.58853	28.274	28.274	28.272
29	28.274	28.274	31.416	0.58853	0.58853	28.274	28.274	28.318
30	31.416	nan	31.416	0.58853	0.58853	40.841	34.558	31.267
31	31.416	nan	31.416	0.58853	0.58853	31.416	31.416	31.406
32	31.416	nan	31.416	0.58853	0.58853	31.416	31.416	31.441
33	0	31.416	31.416	0.58853	0.58853	94.248	9.4247	34.402
34	34.558	nan	31.416	0.58853	0.58853	34.558	34.558	34.535
35	34.558	34.558	37.699	0.58853	0.58853	34.558	34.558	34.569
36	34.558	37.699	37.699	0.58853	0.58853	28.274	31.416	34.708
37	37.699	37.699	37.699	0.58853	0.58853	37.699	37.699	37.659
38	37.699	nan	37.699	0.58853	0.58853	37.699	37.699	37.703
39	37.699	nan	37.699	0.58853	0.58853	34.558	37.699	37.835
40	40.841	50.265	37.699	0.58853	0.58853	40.841	40.841	40.779
41	40.841	40.841	43.982	0.58853	0.58853	40.841	40.841	40.841
42	40.841	nan	43.982	0.58853	0.58853	40.841	40.841	40.956
43	43.982	43.982	43.982	0.58853	0.58853	43.982	43.982	43.897
44	43.982	43.982	43.982	0.58853	0.58853	43.982	43.982	43.982
45	43.982	43.982	43.982	0.58853	0.58853	43.982	43.982	44.073
46	47.124	nan	43.982	0.58853	0.58853	47.124	47.124	47.015
47	47.124	47.124	43.982	0.58853	0.58853	47.124	47.124	47.124
48	47.124	nan	50.265	0.58853	0.58853	47.124	47.124	47.191
49	50.265	nan	50.265	0.58853	0.58853	78.54	53.407	50.135
50	50.265	nan	50.265	0.58853	0.58853	50.265	50.265	50.263
51	50.265	50.265	50.265	0.58853	0.58853	50.265	50.265	50.31
52	53.407	nan	50.265	0.58853	0.58853	43.982	56.549	53.259
53	53.407	53.407	50.265	0.58853	0.58853	53.407	53.407	53.398
54	53.407	53.407	56.549	0.58853	0.58853	53.407	53.407	53.433
55	0	nan	56.549	0.58853	0.58853	100.53	18.85	56.393
56	56.549	nan	56.549	0.58853	0.58853	56.549	56.549	56.527
57	56.549	nan	56.549	0.58853	0.58853	56.549	56.549	56.561
58	56.549	nan	56.549	0.58853	0.58853	50.265	52.111	56.7
59	59.69	59.69	56.549	0.58853	0.58853	59.69	59.69	59.652
60	59.69	59.69	62.832	0.58853	0.58853	59.69	59.69	59.694
61	59.69	65.973	62.832	0.58853	0.58853	56.549	59.69	59.827
62	62.832	62.832	62.832	0.58853	0.58853	62.832	62.832	62.772
63	62.832	nan	62.832	0.58853	0.58853	62.832	62.832	62.832
64	62.832	62.832	62.832	0.58853	0.58853	53.407	62.832	62.948
65	65.973	nan	62.832	0.58853	0.58853	65.973	65.973	65.89
66	65.973	65.973	69.115	0.58853	0.58853	65.973	65.973	65.973
67	65.973	59.69	69.115	0.58853	0.58853	65.973	65.973	66.066
68	69.115	69.115	69.115	0.58853	0.58853	69.115	69.115	69.007
69	69.115	69.115	69.115	0.58853	0.58853	69.115	69.115	69.115
70	69.115	69.115	69.115	0.58853	0.58853	69.115	69.115	69.184
71	72.257	nan	69.115	0.58853	0.58853	78.54	81.681	72.127
72	72.257	nan	69.115	0.58853	0.58853	72.257	72.257	72.254
73	72.257	72.257	75.398	0.58853	0.58853	72.257	72.257	72.303
74	75.398	nan	75.398	0.58853	0.58853	81.681	81.681	75.251
75	75.398	nan	75.398	0.58853	0.58853	75.398	75.398	75.389
76	75.398	nan	75.398	0.58853	0.58853	75.398	75.398	75.426
77	0	75.398	75.398	0.58853	0.58853	109.96	177.57	78.384
78	78.54	78.54	75.398	0.58853	0.58853	78.54	78.54	78.519
79	78.54	78.54	81.681	0.58853	0.58853	78.54	78.54	78.553
80	78.54	81.681	81.681	0.58853	0.58853	78.54	666.02	78.692
81	81.681	81.681	81.681	0.58853	0.58853	81.681	81.681	81.644
82	81.681	nan	81.681	0.58853	0.58853	81.681	81.681	81.686
83	81.681	nan	81.681	0.58853	0.58853	78.54	81.681	81.819
84	84.823	nan	81.681	0.58853	0.58853	84.823	84.823	84.765
85	84.823	84.823	87.965	0.58853	0.58853	84.823	84.823	84.824
86	84.823	nan	87.965	0.58853	0.58853	144.51	84.823	84.941
87	87.965	87.965	87.965	0.58853	0.58853	87.965	87.965	87.883
88	87.965	87.965	87.965	0.58853	0.58853	87.965	87.965	87.965
89	87.965	87.965	87.965	0.58853	0.58853	87.965	87.965	88.059
90	91.106	nan	87.965	0.58853	0.58853	91.106	91.106	91
91	91.106	91.106	87.965	0.58853	0.58853	91.106	91.106	91.106
92	91.106	87.965	94.248	0.58853	0.58853	91.106	91.106	91.176
93	94.248	94.248	94.248	0.58853	0.58853	97.389	94.647	94.119
94	94.248	nan	94.248	0.58853	0.58853	94.248	94.248	94.246
95	94.248	94.248	94.248	0.58853	0.58853	94.248	94.248	94.295
96	97.389	94.248	94.248	0.58853	0.58853	100.53	103.67	97.243
97	97.389	97.389	94.248	0.58853	0.58853	97.389	97.389	97.381
98	97.389	97.389	100.53	0.58853	0.58853	97.389	97.389	97.418
99	0	nan	100.53	0.58853	0.58853	-2543.7	12.709	100.38
\end{lstlisting}
\newpage
可以看到二分法在所有范围内可以稳定输出所有零点(结果中的部分0是因为在给定区间范围内没有零点或者有两个零点,适当调整程序使得可以调整收敛范围即可消除)。而弦割、牛顿、马勒法虽然都可以收敛得解,但是得到的却并不一定是距离最近的解,原因是切线、割线的延长线可能到达别的解的区域,本质上是由于距离真解不够近造成的,是我们输出解的方式所决定的。第一个函数采用的不动点法较为优秀的求出了距离近的根,而史蒂芬孙法会有部分bug出现。而第二个函数由于其周期性,只会收敛到第一个零点。最终以列表的形式给出优缺点。
\begin{center}
\begin{table}[]
\begin{tabular}{|c|c|c|}
\hline
名称   & 优点           & 缺点                  \\ \hline
二分法  & 收敛范围大,应用广泛   & 收敛速度慢               \\ \hline
史蒂芬孙 & 收敛速度快        & 对函数有很高的要求           \\ \hline
不动点  & 收敛速度快;迭代选择多样 & 对函数有很高的要求;选择过多易纠结   \\ \hline
牛顿   & 收敛速度快        & 对函数有较高的要求;需要求出函数的导数 \\ \hline
弦割   & 收敛速度较快,应用范围广 & 优势不突出               \\ \hline
马勒   & 收敛速度快,应用范围广 & 运算步骤多               \\ \hline
\end{tabular}
\caption{优缺点对比}
\end{table}
\end{center}

\subsection{新算法}
考虑非线性方程$$f(x)=0$$在$x_{0}$处做泰勒展开$$f ( x ) = f \left( x _ { 0 } \right) + f ^ { \prime } \left( x _ { 0 } \right) \left( x - x _ { 0 } \right) + \frac { f ^ { \prime \prime } \left( x _ { 0 } \right) } { 2 ! } \left( x - x _ { 0 } \right) ^ { 2 } + \cdots$$取其二次近似有
$$f \left( x _ { 0 } \right) + f ^ { \prime } \left( x _ { 0 } \right) \left( x - x _ { 0 } \right) + \frac { f ^ { \prime \prime } \left( x _ { 0 } \right) } { 2 ! } \left( x - x _ { 0 } \right) ^ { 2 }\approx 0$$
若$f''(x_{0})\neq 0$且$\sqrt{f'(x_{0})^{2}-2f(x_{0})f''(x_{0})} \geq 0$,则有
$$x=x_{0}+\frac{-f'(x_{0}) \pm \sqrt{f'(x_{0})^{2}-2f(x_{0})f''(x_{0})}}{2f''(x_{0})}$$特别的,$\pm$取$sign(f'(x_{0}))$。所以,可以得到迭代序列
$$x_{k+1}=x_{k}+\frac{-f'(x_{k}) \pm \sqrt{f'(x_{k})^{2}-2f(x_{k})f''(x_{k})}}{2f''(x_{k})}$$
并且,可以看出如果$f ^ { \prime \prime } \left( x _ { k } \right)\neq 0$那么令$$g(x)=x+\frac{-f'(x) \pm \sqrt{f'(x)^{2}-2f(x)f''(x)}}{2f''(x)}$$则方程$$g(x)=0$$和下面一系列方程等价。
$$\frac{-f'(x) \pm \sqrt{f'(x)^{2}-2f(x)f''(x)}}{2f''(x)}=0$$
$$-f'(x) \pm \sqrt{f'(x)^{2}-2f(x)f''(x)}=0$$
$$2f(x)f''(x)=0$$
$$f(x)=0$$
而经过求导运算以后,发现在p的一个小临域内,有g'(x)连续,所以该算法是收敛的。下面编写程序输出迭代过程和结果。程序见附录C,结果如下。
\begin{lstlisting}
0.6
0.59712657
0.5949744028
0.5933619055
0.5921534455
0.5912476131
0.5905685265
0.5900593733
0.5896775993
0.5893913199
0.589176639
0.5890156445
0.5888949077
0.5888043602
0.5887364525
0.5886855233
0.5886473273
0.5886186808
0.5885971962
0.588581083
0.5885689981
0.5885599345
0.5885531368
0.5885480386
0.5885442149
0.5885413472
0.5885391964
0.5885375833
0.5885363735
0.5885363735
\end{lstlisting}
可以看出,这种算法虽然收敛,然而速度非常缓慢,计算也异常麻烦,而且对函数还有二阶导的要求。所以,缺点非常明显,相比于课上的算法应当感到羞愧。课上提出的算法不愧是被前任检验许久!
\begin{appendices}
      \section{代码一}
      \begin{lstlisting}
#include <iostream>
#include <iomanip>
#include <math.h>
#define Epsilon 0.000001
using namespace std;
int sign(double x){
    if(x>0){return 1;}
    else if(x==0){return 0;}
    else{return -1;}
}
double f(double x){
    return exp(-x)-sin(x);
}
double h(double x){
    return exp(-x)-sin(x)+x;
}
double g(double x){
    return asin(exp(-x));
}
double half(double x){
    double m=x;
    double a=m-1.5,b=m+1.5;
    if(sign(f(a))==sign(f(b))){return 0;}
    else{
        m=a+(b-a)/2;
        if(f(m)==0){return m;}
        else{
        while((a-b)*sign(a-b)>Epsilon){
        if(sign(f(m))==sign(f(a))){a=m;}
        else{b=m;}
            m=a+(b-a)/2;
        }
            return m;
        }
    }
}
double steffen1(double x){
    double x1=0,x2=x;
    while((x1-x2)*sign(x1-x2)>Epsilon){
        x1=x2;
        x2=x1-(h(x1)-x1)*(h(x1)-x1)/(h(h(x1))-2*h(x1)+x1);
    }
        return x2;
    }
double fixedpoint1(double x){
    double x1=0,x2=x;
    while((x1-x2)*sign(x1-x2)>Epsilon){
        x1=x2;
        x2=h(x2);
    }
    return x2;
}
double steffen2(double x){
    double x1=0,x2=x;
    while((x1-x2)*sign(x1-x2)>Epsilon){
        x1=x2;
        x2=x1-(g(x1)-x1)*(g(x1)-x1)/(g(g(x1))-2*g(x1)+x1);
    }
    return x2;
}
double fixedpoint2(double x){
    double x1=0,x2=x;
    while((x1-x2)*sign(x1-x2)>Epsilon){
        x1=x2;
        x2=g(x2);
    }
    return x2;
}
double newton(double x){
    double x1=0,x2=x;
    while((x1-x2)*sign(x1-x2)>Epsilon&&f(x1)!=0){
        x1=x2;
        x2=x1-(exp(-x1)-sin(x1))/(-exp(-x1)-cos(x1));
    }
    return x1;
}
double line(double x){
    double temp,x1=x-0.1,x2=x;
    while((x1-x2)*sign(x1-x2)>Epsilon&&f(x1)!=0){
        temp=x2-f(x2)*(x2-x1)/(f(x2)-f(x1));
        x1=x2;
        x2=temp;
    }
    return x1;
}
double muller(double x0,double x1,double x2){
    double p,h,a,b,d,e,sigma1,sigma2,h1=x1-x0,h2=x2-x1;
    sigma1=(f(x1)-f(x0))/h1;sigma2=(f(x2)-f(x1))/h2;
    a=(sigma2-sigma1)/(h1+h2);
    while(1){
        b=sigma2+h2*a;
        d=sqrt(b*b-4*f(x2)*a);
        if(sign(b-d)*(b-d)<sign(b+d)*(b+d)){e=b+d;}
        else{e=b-d;}
        h=-2*f(x2)/e;
        p=x2+h;
        if(h*sign(h)>Epsilon)break;
        x0=x1;
        x1=x2;
        x2=p;
        h1=x1-x0;
        sigma1=(f(x1)-f(x0))/h1;
        sigma2=(f(x2)-f(x1))/h2;
        a=(sigma2-sigma1)/(h2+h1);
    }
    return p;
}
int main() {
    //for(double x=1;x<100;x=x+1){
        //cout<<setprecision(5)<<x<<"\t"<<half(x)<<"\t"<<steffen1(x)<<"\t"<<
       // fixedpoint1(x)<<"\t"<<steffen2(x)<<"\t"<<fixedpoint2(x)<<"\t"<<
        //newton(x)<<"\t"<<muller(x)<<"\t"<<endl;
        
    //}
    cout<<setprecision(10)<<"\t"<<half(0.6)<<"\t"<<steffen1(0.6)<<endl;
    cout<<fixedpoint1(0.6)<<steffen2(0.6)<<"\t"<<fixedpoint2(0.6)<<endl;
    cout<<newton(0.6)<<"\t"<<line(0.6)<<"\t"<<endl;
    return 0;
}


      \end{lstlisting}
 \section{代码二}
 \begin{lstlisting}
 setwd('/Users/mayuheng/Desktop')
data<-read.table('2.txt',stringsAsFactors = FALSE)
index<-c(0,which(grepl('T',data[[1]])),length(data[[1]])+1)
result<-list(NULL)
for(i in 1:(length(index)-1)){
  print(i)
  result[[i]]<-c(data[[1]][(index[i]+1):(index[i+1]-1)],
  rep(" ",length=40-index[i+1]+index[i]+1))
}
write.csv(result,file='result.csv')
write.table(result,file='result.txt')
 \end{lstlisting}
  \section{代码三}
 \begin{lstlisting}
 #include <iostream>
#include <iomanip>
#include <math.h>
#define Epsilon 0.000001
using namespace std;
int sign(double x){
    if(x>0){return 1;}
    else if(x==0){return 0;}
    else{return -1;}
}
 double m(double x){
    return (-cos(x)-exp(-x)+sqrt(2*cos(x)*exp(-x)+1))/2/(-sin(x)-exp(-x));
}
double New(double x){
    double x1=0,x2=x;
    while((x1-x2)*sign(x1-x2)>Epsilon&&f(x1)!=0){
        x1=x2;
        x2=x1+m(x1);
        cout<<x1<<endl;
    }
    return x1;
}
int main() {
    cout<<setprecision(10)<<New(0.6);
    return 0;
}
  \end{lstlisting}
\end{appendices}
\end{document}
